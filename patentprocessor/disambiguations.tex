One of the primary problems with conducting meaningful research with
USPTO patent data is the high variability in quality. Cities are misspelled
or mislisted. Organizations are alternatively abbreviated and listed
in full with little modicum of consistency. Inventors, lawyers and
assignees will misspell their names, change their names and unpredictably
list their middle initials or names. The Berkeley patent database
provides facilities to account for these errors, and codifies the
disambiguation of such records in order to make possible their accurate
retrieval.


\subsection{Geocoding}

There are over 12 million locations listed in the USPTO patent weekly
downloads from 1975 to 2013, with 350,000 unique tuples of \verb`(city, state, country)`.
These tuples follow the typical motif of data problems in the rest
of the patent data: incorrect or nonstandard country codes, inconsistent
romanization of foreign locations and various misspellings. We resolved
the ambiguities in the location data using a propietary disambiguation
technique developed by Google. When new patent data is processed,
we run a series of data cleaning processes to correct for some of
the common errors, then cross reference with the lookup table~\cite{geotable}
obtained through the Google disambiguation.

A detailed analysis of the problems with USPTO location data and our
handling of locations can be found through a related Fung Institute
publication~\cite{geocoding}.

Locations are associated with assignees, inventors and lawyers. Typically,
a patent record's ``location'' is the location of the first inventor
listed on the patent.


\subsection{Assignees}

For a given patent, the assignees are the entities (either organizations
or individuals) that have property rights to the patent. The assignee
records are imperative for firm-level analysis of patent data, and
are used for tracking ownership of patents. The weekly releases of
patent documents only contain the original assignee of a patent when
it was initially granted.

However, it is difficult to obtain accurate results for simple (and
necessary) questions such as \emph{``which patents are owned by firm
X?''} because of the pandemic inconsistency of spellings. A cursory
search for assignee records that resemble General Electric yields
the following:
\begin{itemize}
\item General Electric Company 
\item General Electric 
\item General Electric Co.. 
\item General Electric Capital Corporation 
\item General Electric Captical Corporation 
\item General Electric Canada 
\item General Electrical Company 
\item General Electro Mechanical Corp 
\item General Electronic Company 
\item General Eletric Company 
\end{itemize}
This is not even a complete list of all the (mis)representations of
General Electric, but already we can see the potential issues with
trying to get accurate results.

We do not yet provide fully featured entity resolution for assignee
records, but we do maintain a preliminary disambiguation of the records
that corrects for minor misspellings. We do this by applying the Jaro-Winkler~\cite{jw}
string similarity algorithm to each pair of raw assignee records.
Two records that are within a certain bound of similarity are considered
the same, and are linked together.

It is not tractable to perform pairwise computation on each of the
5,850,531 raw assignee records in the database (at time of writing),
so we group the assignees by their first letter, and then perform
the pairwise comparisions within each of these blocks. This allows
us to hold a smaller chunk of the assignees in memory at each step,
and achieves near the same accuracy.


\subsection{Lawyers}

The raw lawyer records follow much of the same deficiencies in quality
as the assignee records. Again, we only offer a preliminary disambiguation
of lawyer records using the same algorithm as described above, but
future development will yield more accurate results.


\subsection{Inventors}

We provide a polished disambiguation mechanism for inventor records.
Using the published name of an inventor, the patent technology class,
co-inventor names, published location and original assignee, we are
able to infer with more than 95\% accuracy which inventor records
are the same across all records in the patent database.

A detailed summary of our technique can be found through a related
Fung Institute publication~\cite{newdisambiguation}. 
