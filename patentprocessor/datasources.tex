The unified patent dataset is composed of processed data from two
separate sources: the Harvard Dataverse Network (DVN)~\cite{disambiguation}
collection of patent data from 1975 through 2010 and the weekly distributions
of Google-hosted USPTO records~\cite{googlefiles}\cite{googlefiles-applications}.


\subsection{Harvard DVN}

The Harvard DVN patent database consists of data drawn from the National
Bureau of Economic Research (NBER), weekly distributions of patent
data from the USPTO, and to a small extent, the 1998 Micropatent CD
product~\cite{micropatent}. The schema of the database can be found
in the Appendix.

While the Harvard DVN patent database was, prior to the UC Berkeley
patent database, the most extensively complete amalgamation of United
States patent data, it is not without its problems. Firstly, there
is little information as to the actual meanings of the columns in
the databases. Without sufficient prior knowledge of patent structure,
it is difficult to glean the semantic significance of each column.
The names alone are often abbreviated and hard to discern. Secondly,
because the DVN database is a combination of several sources into
a single database schema, certain patent entries from NBER and Micropatent
are incomplete where their data source did not provide all the requisite
data. The data obtained from the weekly distributions suffers from
being made available in several different formats. The parser that
was developed to handle the data is overly complicated and does not
handle edge cases well, resulting in missing patent metadata where
the parser did not account for a subtle change in format~\cite{oldparser}.
This is analyzed in greater detail below.


\subsection{USPTO Weekly Distributions}

\begin{table*}[t]
\center %
\begin{tabular}{|l|l|}
\hline 
Time Span  & Data Format \tabularnewline
\hline 
-1974  & paper-based \tabularnewline
1975  & unknown. Data obtained from Micropatent \tabularnewline
1976 - 2001  & Green Book (CITE) APS key-value \tabularnewline
2001  & SGML ST. 32 v2.4 \tabularnewline
2002 - 2004  & Red Book (CITE) XML ST. 32 v2.5 \tabularnewline
2005  & Red Book XML ST. 36 (ICE) v4.0 \tabularnewline
2006  & Red Book XML ST. 36 (ICE) v4.1 \tabularnewline
2007 - 2012  & Red Book XML ST. 36 (ICE) v4.2 \tabularnewline
2013  & Red Book XML ST. 36 (ICE) v4.3 \tabularnewline
2013 -  & Red Book XML v4.4 \tabularnewline
\hline 
\end{tabular}\caption{Table of USPTO grant data formats}


\label{fig:grantformats} 
\end{table*}
\begin{table*}[t]
\center %
\begin{tabular}{|l|l|}
\hline 
Time Span  & Data Format \tabularnewline
\hline 
-2001 & paper-based \tabularnewline
2001  & SGML ST. 32 v1.5\tabularnewline
2002 - 2004  & Red Book (CITE) XML ST. 32 v1.6 \tabularnewline
2005  & Red Book XML ST. 36 (ICE) v4.1 \tabularnewline
2006 - 2012 & Red Book XML ST. 36 (ICE) v4.2 \tabularnewline
2013 -  & Red Book XML v4.3\tabularnewline
\hline 
\end{tabular}\caption{Table of USPTO grant data formats}


\label{fig:applicationformats} 
\end{table*}


The USPTO distributions take the form of zip archives containing concatenated
XML (Extensible Markup Language) documents, each of which contains
the full text of each patent grant and patent application issued every
week. Prior to 1975, the USPTO used a purely paper-based system before
transitioning to a raw-text key-value and later SGML-based key-value
store %
\footnote{Standard Generalized Markup Language%
}. Patent documents were made available in the XML format starting
in 2001. Although this data is made freely available, the fact that
digital USPTO patent data spans eight different formats and occupies
more than 70 GB (compressed) over the 37 years of its existence makes
rendering the data into an amenable form a nontrivial problem (see
Table~\ref{fig:grantformats} and Table~\ref{fig:applicationformats}).
Patent application data, though only available in a digital format
back to 2001, is nonetheless available in six different formats~\cite{xmlresources}~\cite{xmlretrospective}.
